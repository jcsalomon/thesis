% !TEX TS-program = xelatex 
% !TEX encoding = UTF-8
\documentclass{article}

\usepackage{ifxetex}

% AMS Math
\usepackage{amsmath}

\ifxetex	% XɘLᴬTeX & fonts
	\usepackage{fontspec, xunicode, xltxtra, unicode-math}
	\setmainfont[Mapping=tex-text,Numbers={OldStyle,Proportional}]{Constantia}
	\setsansfont[Mapping=tex-text,Numbers={OldStyle,Proportional}]{Candara}
	\setmonofont[Numbers=OldStyle]{Consolas}
	\setmathfont{Cambria Math}
\else
	\usepackage[utf8]{inputenc}
	\let\mathbfup\mathbf
\fi

% Matrices, vectors, functions, etc.
\usepackage{mlist}

% Abbreviations & punctuation
\usepackage{xspace}
\usepackage[abbreviations]{foreign}
\defasforeign*[adastra]{Sic itur ad astra}	% Virgil, Aeneid book IX, line 641
\ifxetex
	\DeclareRobustCommand\dash{%
		\ifvmode\leavevmode\else\unskip\nobreak\thinspace\fi
		\textemdash\thinspace\ignorespaces
	}
	\catcode`\—=\active
	\let—=\dash
	\newcommand{\No}{№\xspace}
\else
	\newcommand{\No}{No\xperiod}
\fi
\makeatletter
	\let\old@ampersand\&
	\renewcommand{\&}{\textit{\old@ampersand}}
\makeatother

% Date & time formatting
\usepackage{datetime}
\newdateformat{usvardate}{%
	\dayofweekname{\THEDAY}{\THEMONTH}{\THEYEAR} %
	\monthname[\THEMONTH] \ordinaldate{\THEDAY}, %
	\THEYEAR}
\usvardate
\newtimeformat{scampmtime}{%
	\THEHOURXII\timeseparator\twodigit{\THEMINUTE}~\textsc{\amorpmname}}
\settimeformat{scampmtime}

% Bibliography
\usepackage[hyperref, style=authoryear]{biblatex}
\bibliography{control}
\renewcommand{\subtitlepunct}{\ifterm{}{\addcolon}\space}
\DeclareCiteCommand*{\citetitle}
	{\boolfalse{citetracker}%
		\boolfalse{pagetracker}%
		\usebibmacro{prenote}}
	{\indexfield{indextitle}%
		\printtext[citetitle]{%
		\printfield[noformat]{title}%
		\iffieldundef{subtitle}{}{\subtitlepunct}%
		\printfield[noformat]{subtitle}}}%
	{\multicitedelim}
	{\usebibmacro{postnote}}

% Indexing
%\usepackage{makeidx, showidx}
%\makeindex

% Theorems, examples, etc.
\usepackage[amsmath, hyperref, thmmarks]{ntheorem}	%, thref
\theoremstyle{plain}
\theoremsymbol{}
\theoremheaderfont{\scshape}
\theoremseparator{:}
\theorembodyfont{\upshape}

\newtheoremstyle{example}%
	{\item[\hskip\labelsep \textsc{##1 ##2}:]}%
	{\item[\hskip\labelsep \textsc{##1 ##2} (\textit{##3}):]}
\newtheoremstyle{nonumberexample}%
	{\item[\hskip\labelsep \textsc{##1}:]}%
	{\item[\hskip\labelsep \textsc{##1} (\textit{##3}):]}
\theoremstyle{example}
\newtheorem*{ex}{Example}
\newtheorem{defn}{Definition}[section]

% Cross-references
\usepackage[letterpaper, debug]{hyperref}

\usepackage{cleveref}
\crefname{ex}{Example}{examples}
\crefname{defn}{Definition}{definitions}


% Local math stuff
\newcommand*{\fl}{{\mathrm{f}}}
\newcommand*{\il}{0}	%{{\mathrm{i}}}
\newcommand*{\veca}{{\mathbf{a}}}
\newcommand*{\vecu}{{\mathbf{u}}}
\newcommand*{\vecx}{{\mathbf{x}}}

\newcommand*{\intt}[1]{\int_{t_\il}^{t_\fl}\!\!#1}

\newcommand*{\textinterject}[1]{\text{\enspace#1\enspace}}

% Differential d
\newcommand*{\diffd}{d}	% ⅆ
\makeatletter
	% Give \dd the same height whether followed by an exponent or no.
	\newcommand*{\dd}{\@ifnextchar^{\jcs@dd}{\jcs@dd^{}}}
	% Give \jcs@dd effective height of \mathstrut,
	% 	operator spacing on left & funky spacing code on right.
	\def \jcs@dd^#1{\mathop{\diffd\mathstrut}\nolimits^{#1}\jcs@dd@eatspace}
	% Look at what follows the \dd, then put space (or not).
	\def \jcs@dd@eatspace{\futurelet\jcs@dd@arg\jcs@dd@putspace}
	% Put space, or don’t, depending on whether next token is brace or no
	\def \jcs@dd@putspace{%
		\def \@jcs@dd@space {\!\!}%
		\ifx \jcs@dd@arg(%
			\let \@jcs@dd@space \!
		\else\ifx \jcs@dd@arg[%
			\let \@jcs@dd@space \!
		\else\ifx \jcs@dd@arg\{%
			\let \@jcs@dd@space \!
		\fi\fi\fi\@jcs@dd@space}
\makeatother

\begin{document}
\title{Control Systems for the\\%
%	Northrop Grumman\\%
	Lunar Lander X-Prize Challenge\\%
	Interim Report \No 1:\\%
	Formulation of the Problem}
\author{Joel C. Salomon}
\date{\today\\\currenttime}
\special{pdf: docinfo <<
	/Title		(Thesis Report 1: Formulation of the Problem)
\ifxetex /Creator	(XɘLᴬTeX with hyperref package)\else\fi
	/Author	(Joel C. Salomon)
	/CreationDate	(D:20090527214601)
	/ModDate	(D:\pdfdate)
	/Subject	(Engineering)
	/Keywords	(control systems;optimal control;rocket landing)
	>>}
\maketitle

\begin{abstract}
	“The objective of optimal control theory is \emph{
	to determine the control signals that will cause a process to
	satisfy the physical constraints and at the same time
	minimize \textup{(}or maximize\textup{)} some performance criterion}.”%
	~\parencite{kirk:optintro}
\end{abstract}


\section*{Overview}
The basic plan for this thesis project has three main steps:
\begin{enumerate}
	\item Work through the math for the soft-landing problem;
	\item Ditto for the attitude control problem;
	and, if I have enough time,
	\item Unify the two problems, probably using Geometric Algebra.
\end{enumerate}
Which all \emph{sounds} easy enough….

The basic text I’m using is Donald E. Kirk’s \citetitle*{kirk:optintro}
from the Prentice-Hall Networks series~\parencite{kirk:optintro}.
Also useful are Stanley Shinners’s
\citetitle{shinners:modern}~\parencite{shinners:modern}
and \citetitle{shinners:advanced}~\parencite{shinners:advanced}.

This report is mainly going to describe the problems I wish to solve for my thesis:
attitude and position control for a rocket lander.
Description of actual \emph{work} toward this goal will come in a future report.


\cleardoublepage\tableofcontents


\section{Modeling the System}\label{sec:model}
There are three main items that must be included in a mathematical model of a system
to formulate an optimal control problem:
the \hyperref[sec:statespace]{system \emph{dynamics}},
the \hyperref[sec:constraints]{\emph{constraints} on system state},
and a \hyperref[sec:criterion]{\emph{performance criterion}}.
The language we use for this is that of \emph{state space}.


\subsection[State Space Dynamics]{The State Space Model of System Dynamics}
	\label{sec:statespace}
As mentioned, we use the language of state space to model
the process which we wish to control.

If we define the functions % \(x_1(t)\), \(x_2(t)\), …, \(x_n(t)\)
\[
	x_1(t), x_2(t), …, x_n(t)
\]
as the \emph{state variables} of the process at time \( t \),
and % \(u_1(t)\), \(u_2(t)\), …, \(u_m(t)\)
\[
	u_1(t), u_2(t), …, u_m(t)
\]
as the \emph{control inputs} to the process at time \( t \),
then we can describe the dynamics of the system
by a set of \( n \) first-order differential equations —
the \emph{state equations}:
\begin{equation} \label{eq:stateeq}
\begin{split}
	\dot{x}&_1(t) = \func{a_1}{
		x_1(t), x_2(t), …, x_n(t),
		u_1(t), u_2(t), …, u_m(t), t } \\
	\dot{x}&_2(t) = \func{a_2}{
		x_1(t), x_2(t), …, x_n(t),
		u_1(t), u_2(t), …, u_m(t), t } \\
	{} & \vdots \\
	\dot{x}&_n(t) = \func{a_n}{
		x_1(t), x_2(t), …, x_n(t),
		u_1(t), u_2(t), …, u_m(t), t }.
\end{split}
\end{equation}
Each \( \dot{x}_i(t) \) is a — possibly non-linear —
function \( a_i \) of the states, the control inputs, and time.

\begin{ex}
	For the single-axis soft-landing problem,
	the relevant states of the craft are its mass and vertical position and velocity
	and the control is the fuel flow to the rocket engine.
	The dynamics of the system are the result of thrust working against gravity.
\end{ex}
\begin{ex}
	For the attitude-control problem,
	the states are the craft’s attitude, moment of inertia, and rotational velocity.
	(A currently-open issue is how best to represent these quantities.)
\end{ex}

For convenience’s sake, we prefer the tools of linear algebra
to the set of equations in~\eqref{eq:stateeq} above:
\begin{defn}[State \& Control Vectors]
	We define the vectors
	\[
		\vecx(t) \triangleq
		\begin{bmatrix}
			x_1(t) \\ x_2(t) \\ \vdots \\ x_n(t)
		\end{bmatrix}
		\textinterject{and}
		\vecu(t) \triangleq
		\begin{bmatrix}
			u_1(t) \\ u_2(t) \\ \vdots \\ u_m(t)
		\end{bmatrix};
	\]
	we call \( \vecx(t) \) the \emph{state vector}
	and  \( \vecu(t) \) the \emph{control vector}
	for the system at time \( t \).
\end{defn}
The state equations can now be rewritten as
\begin{equation}
	\label{eq:stateeq:vect}
	\tag{\ref*{eq:stateeq}${}′$}
	\dot{\vecx}(t) =
		\func{\veca}{\vecx(t), \vecu(t), t}
\end{equation}
where the definition of \( \veca \) in \eqref{eq:stateeq:vect}
is apparent by comparison to \eqref{eq:stateeq}.

Over the time interval \( t \in [t_\il, t_\fl] \), for systems as just described,
the following definitions hold:
\begin{defn}[State \& Control Trajectories]
	A history of control-input values  \( \vecu(t) \)
	over the interval \( [t_\il, t_\fl] \)
	is denoted by \( \vecu \)
	and is called a \emph{control history}, or simply a \emph{control}.
	\par\noindent
	A history of state values  \( \vecx(t) \)
	over the interval \( [t_\il, t_\fl] \)
	is called a \emph{state trajectory}
	and is denoted by \( \vecx \).
\end{defn}\noindent
The terms “history”, “curve”, and “trajectory” are interchangeable.
Note carefully, however, the distinction between
the state \emph{trajectory} \( \vecx \)
and the instantaneous \emph{value} \( \vecx(t) \) of the state vector
at some given time \( t \).


\subsection{Modeling Physical Constraints}\label{sec:constraints}
There will often be constraints on the states or the control inputs.
These may be physical in nature
(\eg, that the altitude of a rocket must always be greater than zero
or that the rate of mass-change due to fuel-flow must always be negative)
or regulatory (\eg, altitude or power limits).

The usual way of indicating these constraints is
with a set of predicates (often inequalities) on state or input variables,
\eg, $| x_1(t) | < 2$.

\subsection{Specifying a Performance Criterion}\label{sec:criterion}
Obviously, if a control system is to be “optimal”,
there must be some measure of performance to be selected for.

In classical control theory, the measure is often the system response
to step or ramp input, or a frequency response characteristic.
More generally, we define a performance measure
\begin{equation}
	\label{eq:measure}
	J = \func{h}{\vecx(t_\fl),t_\fl} +
		\intt{\func{g}{\vecx(t), \vecu(t), t} \dd t}
\end{equation}
for some scalar functions $h$ \& $g$, then attempt to minimize $J$.
The choice of $h$ \& $g$ usually has some physical motivation.
\begin{ex}[Minimum-Time Problems]
	If the goal of the controller is to transfer the system
	from an arbitrary initial state $\vecx(t_\il) = \vecx_\il$
	to a specified target space $S$ in minimum time,
	then the performance measure to be minimized is
	\[
		J = t_\fl - t_\il = \intt{\dd t},
	\]
	with $t_\fl$ defined as the first instant when $\vecx(t) \in S$.
\end{ex}\noindent
Other common performance measures are minimum-final-deviation from a reference, \ie,
\[
%	J =	\left‖ \mathbf{r}(t_\fl) - \vecx(t_\fl) \right‖^2 =
	J =	‖ \mathbf{r}(t_\fl) - \vecx(t_\fl) ‖^2 =
		\sum_{i=1}^n \left[ x_i(t_\fl) - r_i(t_\fl) \right]^2,
\]
and minimum-control-effort, \ie,
\[
	J = \intt{u(t) \dd t}
		\textinterject{or}
	J = \intt{u^2(t) \dd t}
\]
for fuel or energy use, respectively.

\cleardoublepage\printbibliography[heading=bibintoc]

\end{document}
